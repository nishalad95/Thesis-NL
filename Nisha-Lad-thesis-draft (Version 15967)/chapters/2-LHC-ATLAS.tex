\graphicspath{{\subfix{../images/}}}
%-------------------------------------------
%	Chapter 2: LHC and ATLAS
%-------------------------------------------
\doublespacing

%-------------------------------------------
%	Chapter 2: LHC and ATLAS
%-------------------------------------------
\chapter{The Large Hadron Collider and ATLAS Detector}
\label{chapter-2}

Since the completion of its construction in 2008, the \ac{LHC} \cite{Evans:2008zzb} at CERN has extended the frontiers of particle physics through its unprecedented energy and luminosity. Located on the Swiss-French border, the \ac{LHC} is the world’s largest particle accelerator. It is designed to accelerate protons around a 27 km ring until they are travelling just 3 ms$^{-1}$ slower than the speed of light, at which point they are made to collide. The protons travel round the ring 11,000 times per second in two concentric beams, which are guided by superconducting magnets, cooled using liquid helium to \num{-271.3} \si{\degree}C (1.9 K). The beams are made to cross at four locations so that collisions between protons can take place. Around these collision points four specialised detectors, ALICE \cite{AliceCollaboration_2008}, CMS \cite{CMS-TDR-08-001}, LHCb \cite{LHCbCollaboration_2008} and ATLAS \cite{PERF-2007-01}, are located to capture information.

In this chapter, a brief overview of the \ac{LHC} and the accelerator complex at CERN is given in Section \ref{the-lhc}. The ATLAS experiment and an overview of the different detector systems is provided in Section \ref{atlas-section}. Finally, the future of the \ac{LHC} program is presented in Section \ref{hi-lumi}, with a focus on the motivation and challenges that the  \ac{HL-LHC} phase will bring.


\section{The Large Hadron Collider}
\label{the-lhc}
The \ac{LHC} is operated in multi-year runs during which beams of protons travelling in opposite directions are circulated and collide. Between runs there are periods of shutdown, while the accelerator and detector machinery is maintained and upgraded. Run 1 began in 2010 when the \ac{LHC} collided proton bunches, each containing more than $10^{11}$ particles, 20 million times per second, providing 7 TeV proton-proton collisions at instantaneous luminosities of up to 2.1 $\times$ 10$^{32}$ cm$^{−2}$s$^{−1}$.

The centre-of-mass energy was increased to 8 TeV towards the end of Run 1 in 2012. Run-2, which spanned in 2015–2018, further increased the proton-proton collision energy to 13 TeV. During Run-2, the bunch spacing was reduced, leading to a collision rate of 40MHz. Over the course of Run-2, a total usable integrated luminosity of 139 fb$^{−1}$ was recorded. 2022 marked the beginning of Run-3, which, with a higher center of mass energy of 13.6 TeV and peak luminosity at 2 $\times$ 10$^{34}$ cm$^{−2}$s$^{−1}$, is expected to culminate in the approximate tripling of the dataset size. In addition, the number of pp collisions per bunch crossing, referred to collectively as pile-up, is expected to be $\langle \mu \rangle$ = 60-65 at the end of Run-3. A summary of key information about each run is listed in Table \ref{tab:lhc-runs}.

\begin{table}[!htbp]
  \footnotesize\centering
  \setlength{\tabcolsep}{0.5em} % for the horizontal padding
  \begin{tabular}{cc|cccc}
      \toprule
      \textbf{Period} & \textbf{Year} & $\sqrt{s}$ [TeV] 
      & $\langle \mu \rangle$ & \textbf{Bunch spacing} [ns] & \textbf{Luminosity} [cm$^{−2}$s$^{−1}$] \\
      \hline
      Run-1 & 2010--2012 & \SIrange[range-phrase=--,range-units=single,range-exponents=combine]{7}{8}{} & 18 & 50 & $8 \times 10^{33}$ \\
      Run-2 & 2015--2018 & \SI{13  }{} & 34 & 25 & $1\textnormal{--}2 \times 10^{34}$ \\
      Run-3 & 2022--2025 & \SI{13.6}{} & 60-65 & 25 & $2 \times 10^{34}$ \\
      \bottomrule
  \end{tabular}
  \caption{
    Overview of the different \ac{LHC} runs \cite{atlas-lumi-run1,atlas-lumi-run2}.
    The average number of interactions per bunch-crossing is denoted as $\langle \mu \rangle$, and is here averaged over the entire run. Numbers for Run-3 are preliminary and are only provided to give an indication of expected performance.
    % run 1 run 2 trigger https://cds.cern.ch/record/2058218/
    %  2-3 × 1034 cm−2 s−1 and an average pile-up of ~80 collisions/bunch-crossing
    % https://cds.cern.ch/record/2732959/files/LHCP2020_094.pdf
  }
  \label{tab:lhc-runs}
\end{table}

An overview of the accelerator complex at CERN is shown in Fig. \ref{fig: accelerator-complex}. The \ac{LHC} is at the final stage of a chain of accelerators which incrementally step-up the energy of incoming protons. The first accelerator is Linac4, a linear accelerator which accelerates hydrogen atoms to an energy of 160 MeV. Upon leaving Linac4, the hydrogen atoms are stripped of their electrons and the resulting protons are fed into the Proton Synchrotron Booster (PSB), which increases the energy of the protons to 2 GeV. The protons leaving the PSB are passed to the Proton Synchrotron (PS), which increases the energy to 26 GeV, and then from the PS to the Super Proton Synchrotron (SPS), which further increases the energy to 450 GeV. Finally, the proton beams are injected in the \ac{LHC} where they are accelerated to their final energy of 6.5 TeV for Run 2.

\begin{figure}[htb!]
\includegraphics[width=\textwidth]{images/2-LHC-ATLAS/accelerator_complex.pdf}
\caption{An overview of the CERN accelerator complex \cite{CERN:2012:accelerators}. The \ac{LHC} is fed by a series of accelerators starting with Linac4. Next are the Proton Synchrotron Booster, the Proton Synchrotron, and finally the Super Proton Synchrotron which injects protons into the \ac{LHC}.}
\label{fig: accelerator-complex}
\end{figure}



%-------------------------------------------
%	Chapter 2: ATLAS
%-------------------------------------------
\section{The ATLAS Experiment}
\label{atlas-section}

\subsection{The ATLAS Detector}
The ATLAS\footnote[1]{\textbf{A} \textbf{T}oroidal \textbf{L}HC \textbf{A}pparatu\textbf{S}.} detector is one of two general-purpose detectors in operation at the \ac{LHC} \cite{PERF-2007-01}. The experiment aims to make Standard Model precision measurements and test Beyond Standard Model (BSM) theories. In total, the detector is a 44m long cylinder with a diameter of 25m and weighs over 7000 tonnes, shown in Figure \ref{fig: atlas-detector}. The detector’s geometry is cylindrical consisting of a central barrel and two end-caps to ensure forward physics coverage and hermeticity. The ATLAS detector comprises of specialised sub-detectors, orientated coaxially around the nominal interaction point at the centre of the detector. 

\begin{figure}[htb!]
\includegraphics[width=\textwidth]{images/2-LHC-ATLAS/atlas_detector.jpg}
\caption{A 3D model of the entire ATLAS detector \cite{Jon-And:1237407}. Cutouts to the centre of the detector reveal the different sub-detectors which are arranged in concentric layers around the nominal interaction point.}
\label{fig: atlas-detector}
\end{figure}

In order of increasing radial distance, the ATLAS sub-detectors include the inner detector described in Section \ref{inner-detector}, the electromagnetic and hadronic calorimeters, and the outermost muon spectrometer. 

More comprehensive descriptions of the calorimeters and muon spectrometer can be found in the Technical Design Report for the ATLAS detector \cite{inner-detector-TDR}. Since the work in this thesis pertains to tracking, particular attention is given to the Inner Detector (ID) which houses the tracking systems of the ATLAS detector. In Section \ref{coordinate-system}, the coordinate system used at ATLAS and definitions for frequently occurring quantities are also provided.



%-------------------------------------------
%	Chapter 2: The Inner Detector
%-------------------------------------------
\subsection{The Inner Detector}
\label{inner-detector}

The ID system provides high-resolution charged particle trajectory tracking in the range $ \lvert \eta \rvert < 2.5$. The ID is immersed in a 2T axial magnetic field, produced by a superconducting solenoid magnet, which enables the measurement of particle momentum and charge. The inner detector is made up of several sub-systems shown in Figs. \ref{fig:atlas-id-run1} and \ref{fig:atlas-id-run2}. Each sub-system contains specialised hardware and contributes towards a full track reconstruction. 

\begin{figure}[!htbp]
  \centering
  \includegraphics[width=0.75\textwidth]{images/2-LHC-ATLAS/atlas_id.jpg}
  \caption{
    A 3D model of the ATLAS ID, made up of the Pixel and semi-conductor tracker sub-detectors, showing the barrel layers and end-cap disks \cite{atlasid}.
  }
  \label{fig:atlas-id-run1}
\end{figure}

\begin{figure}[!htbp]
  \centering
  \includegraphics[width=0.75\textwidth]{images/2-LHC-ATLAS/atlas_id_xs.png}
  \caption{
    A cross-sectional view of the ATLAS ID, with the radii of the different barrel layers shown \cite{atlastrackingdocs}.
  }
  \label{fig:atlas-id-run2}
\end{figure}

The innermost silicon Pixel detector \cite{pixel} provides high-granularity measurements covering the interaction region and typically provides four spacepoint measurements per track. The silicon Pixel detector comprises four cylindrical barrels at increasing radii from the beamline, and four end-cap disks on each side. The innermost barrel layer is the Insertable B-layer (IBL), which was installed before Run 2 \cite{ATLAS-TDR-19,PIX-2018-001} and lies approximately just 33mm from the beam axis. The second-to-innermost layer is often referred to as the B-layer. The Pixel detector was initially constructed with 80 million readout channels, with the IBL providing an additional 12 million \cite{ibl}. The specification of the Pixel detector determines the impact parameter resolution and the ability to reconstruct primary and secondary vertices. Individual pixels are 50 $\mu$m in the transverse direction $(r,\phi)$ (see Section \ref{coordinate-system} for the ATLAS coordinate system) and 400 $\mu$m in the longitudinal $z$ direction (250 $\mu$m for the IBL). 

The Pixel detector is followed by the Semiconductor Tracker (SCT), which usually provides a further four spacepoint measurements per track. These silicon detectors are complemented by the Transition Radiation Tracker (TRT), which enables radially extended track reconstruction up to $ \lvert \eta \rvert = 2.0$.

The current ID has various limitations that hinder its performance as the \ac{LHC} machine is upgraded. Radiation damage and high detector occupancy result in the requirement for a full replacement of the ID after Run-3 with the new Inner Tracker (ITk) \cite{pileup,itk-strip}. One significant change in the detector layout is that the ITk will consist only of silicon detectors, replacing the TRT, and extend to a 1m radius, whereas the current SCT outer layer extends only to 60cm. The acceptance of the detector will be increased such that that the strip detector covers a range of $ \lvert \eta \rvert = 2.7$ with the pixel detector extending the range to $ \lvert \eta \rvert = 4.0$.

%Outside of the pixel detector, the SCT measures charged particles at an intermediate distance from the collision point and improves the determination of vertex position and track momentum. The SCT consists of four barrel layers and nine end-cap layers on each side. The outermost section of the ID, the TRT, is used for the identification of charged particles and consists of drift tubes that are filled with a mixture of Xe, CO2 and O2, and contain a central gold-plated tungsten wire. When charged particles traverse the TRT, the gas inside the straws is ionised and the free electrons drift towards the wire and are amplified and then read out. In addition, transition radiation provides information on the particle type that passed through the tracker.


%-------------------------------------------
%	Chapter 2: TDAQ
%-------------------------------------------
\subsection{The Trigger}
The 25ns bunch spacing used over the course of Run 2 corresponds to a bunch-crossing or event rate of 40MHz (see Table \ref{tab:lhc-runs}). If the full information for the detector was written out for each event, this would correspond to the generation of 60TB of data each second. This is more than feasibly possible for the read out from the hardware, the processing and storage of the data. This requires the use of a trigger system which quickly makes a decision about whether or not an event is potentially interesting and should be kept for further analysis. The trigger system comprises two levels which search for signs of electrons, muons, taus, photons and jets, as well as events with large total or missing transverse energy. The hardware-based Level-1 (L1) trigger uses coarse information from the calorimeters and muon spectrometer to accept events at an average rate of 100 kHz approximately 2.5 $\mu$s after the event. After the L1 trigger, the software-based High Level Trigger (HLT) makes use of 40,000 CPU cores to make a final selection on surviving events, using full granularity detector information in approximately 200ms. The final event read-out rate is approximately 1.2 kHz, corresponding to 1.2 GBs$^{-1}$ of permanent data storage. More information is provided in \cite{TRIG-2016-01}.


%-------------------------------------------
%	Chapter 2: Inner Detector Trigger
%-------------------------------------------
\subsubsection{The Inner Detector Trigger}

The ability of the ATLAS trigger system to process information from the ID to reconstruct particle trajectories is an essential requirement for the efficient triggering of physics objects. The ID trigger must therefore be able to reconstruct tracks with high efficiency across the entire range of possible physics signatures, as well as handle the input rate of the HLT. This challenge is exacerbated by the very high track and hit multiplicities in the ID that arise from the large pile-up. 

The ID trigger is designed to perform fast online track and vertex reconstruction using measurements from the ID. For Run 2, the ID trigger tracking is performed in two steps; the first algorithm handles trigger-specific pattern recognition and seeded track finding to generate medium quality tracks as quickly as possible. This is collectively known as the \textit{Fast Track Finder} (FTF) algorithm \cite{Penc:2104217, Grandi:2624768}. This step is followed by the \textit{Precision Tracking} (PT), which relies heavily on offline tracking algorithms \cite{T_Cornelissen_2008} improving the track purity and quality by applying tighter requirements. 



%-------------------------------------------
%	Chapter 2: Coordinate system
%-------------------------------------------
\subsection{Coordinate System and Collider Definitions}
\label{coordinate-system}

%% based on https://twiki.cern.ch/twiki/bin/view/AtlasProtected/PubComCommonText

The origin of the coordinate system used by ATLAS is the nominal interaction point in the centre of the detector. As shown in Figure \ref{fig:atlas-coord-system}, the z-axis points along the direction of the beam pipe, while the x-axis points from the interaction point to the centre of the \ac{LHC} ring, and the y-axis points upwards.
The transverse plane lies in $x$-$y$ while the longitudinal direction lies along the z-axis. A cylindrical coordinate system with coordinates ($r$,$\phi$) is used in the transverse plane, where $r$ is the radius from the origin and $\phi$ is the azimuthal angle around the z-axis.

\begin{figure}[!htbp]
  \centering
  \input{images/2-LHC-ATLAS/atlas-coord-system.tex}
  \caption{
    The coordinate system used at the ATLAS detector, showing the locations of the four main experiments located at various points around the \ac{LHC}. The 3-vector momentum $p_{\text{T}} = (p_x, p_y, p_z)$ is shown by the red arrow. Reproduced from \cite{Strong:2020mge}.
  }
  \label{fig:atlas-coord-system}
\end{figure}

%The pseudorapidity is defined in terms of the polar angle $\theta$ as $\eta = -\ln \tan(\theta/2)$.

The polar angle $\theta$ is commonly specified in terms of the pseudorapidity $\eta$, and is defined as
%
\begin{equation}\label{eq:pseudorap}
  \eta = - \ln \left[ \tan \left( \frac{\theta}{2} \right) \right] .
\end{equation}
%
The pseudorapidity is a convenient quantity to work with as differences in $\eta$ are invariant under Lorentz boosts. In addition, particle production is constant as a function of $\eta$.

Additionally, the transverse plane is often used to describe the kinematics of collisions, where the transverse momentum $p_{\text{T}}$ of an object is the projection of its momentum on the transverse plane.

%
\begin{equation}\label{eq:pt}
  p_\text{T} = \sqrt{ {p_x}^2 + {p_y}^2 }
\end{equation}



%-------------------------------------------
%	Chapter 2: Motivation for Hi-Lumi LHC
%-------------------------------------------
\section{Motivation for the HL-LHC}
\label{hi-lumi}

% - the physics motivation to increase the luminosity - brief 

The HL-LHC is an upgrade of the \ac{LHC} to extend its physics reach, particularly in terms of precision measurements in the Higgs sector, by increasing the data collected by an order of magnitude. This will be achieved by increasing the LHC instantaneous luminosity by a factor of up to five compared to the nominal. This will enable the detector's discovery potential and exploration potential to significantly improve. Initially, the luminosity will be increased to $5 \times 10^{34} cm^{−2}s^{−1}$, and subsequently up to $7.5 \times 10^{34} cm^{−2}s^{−1}$ by the mid-2030s.

Since the discovery of the Higgs boson at the ATLAS and CMS experiments \cite{ATLAS-HIGGS, CMS-HIGGS} in 2012, the study of the Higgs sector has greatly expanded to include many precision measurement analyses, predictions from theory and searches for rare production and decay processes. One important question to answer is whether the observed Higgs boson is that predicted from the SM electroweak symmetry breaking mechanism \cite{ewsb} or if it is, in fact, the first signal in some BSM physics. With the accumulated data so far, the identity of the Higgs boson is consistent with Standard Model predictions, but higher-precision measurements could illuminate any potential discrepancies from prediction. Further information on the Higgs mechanism can be found in \cite{Bednyakov_2008}.

%With the accumulated data so far, the identity of the Higgs boson is consistent with SM predictions and all measurements are confined to the couplings of the Higgs to SM particles, which are proportional to the particles’ masses, but higher-precision measurements of these couplings could illuminate any potential discrepancies from prediction.

In general, precision measurements of the Higgs sector provide indirect probes to just about any extension of the Standard Model. There are many theoretical particles predicted by various BSM scenarios that can also be searched for in the HL-LHC. One set of such scenarios falls under the title of Super-Symmetry \cite{supersym}, which predicts super-partners belonging to every fermion and every boson. Direct dark matter searches can also be probed at higher mass scales, and new detector upgrades will facilitate searches for long-lived exotic particles. Additionally, BSM physics can be further probed through rare \textit{b} and \textit{c} hadron decays that may be measured using the increased integrated luminosity \cite{wg-bsm}.

The physics programme offered by the HL-LHC is vast \cite{big-report}; a detailed description of the project and its technological and operational challenges is provided in the HL-LHC Preliminary Design Report \cite{Apollinari:2116337}. The programme would deliver a significant potential for new physics discoveries and incredibly high-precision SM measurements. However, the promising plan for the future is accompanied by many technical challenges that each of the experiments face. The increased luminosity results in greater pile-up, which results in drastically higher detector occupancy and radiation levels. In the case of ATLAS, Figure \ref{fig:pileup-walltime} shows the projected evolution of compute usage from 2020 until 2036 under different R\&D scenarios. The expected pile-up levels during the HL-LHC are around $\langle \mu \rangle$ = 200, demonstrating the need for upgrades to both the detector and algorithmic designs.

\begin{figure}[!htbp]
  \centering
  \includegraphics[width=0.8\textwidth]{images/2-LHC-ATLAS/computing-model.png}
  \caption{
    Projected evolution of compute usage under the conservative (blue) and aggressive (red) R\&D scenarios. The grey hatched shading between the red and blue lines illustrates the range of resource consumption if the aggressive scenario is only partially achieved. The black lines indicate the impact of sustained year-on-year budget increases and improvements in new hardware, that together amount to a capacity increase of 10\% (lower line) and 20\% (upper line). The vertical shaded bands indicate periods during which ATLAS will be taking data \cite{Collaboration:2802918}.
  }
  \label{fig:pileup-walltime}
\end{figure}

%In the case of ATLAS, Figure \ref{fig:pileup-walltime} shows how the reconstruction time per event increases with pile-up. The increase in time is exponential, and the expected pile-up levels during the HL-LHC are around $\langle \mu \rangle$ = 200, demonstrating the need for upgrades to both the detector and algorithmic designs in ATLAS.

%\begin{figure}[!htbp]
%  \centering
%  \includegraphics[width=0.8\textwidth]{images/2-LHC-ATLAS/pileup-walltime.png}
%  \caption{
%    The dependency of reconstruction wall time per event on pileup. The luminosity block count represents short intervals of data taking, in which the instantaneous luminosity is estimated, and from this the integrated luminosity derived \cite{pileup}.
%  }
%  \label{fig:pileup-walltime}
%\end{figure}